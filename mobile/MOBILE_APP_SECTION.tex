\subsection{Мобильное приложение}
\subsubsection{Архитектура мобильного приложения}

Мобильное приложение Leams разработано с использованием \textbf{React Native} и \textbf{Expo}, что позволяет создать кроссплатформенное решение для iOS и Android на основе единой кодовой базы. Это существенно сокращает время разработки и упрощает поддержку приложения.

\begin{figure}[H]
  \centering
  \resizebox{0.8\linewidth}{!}{
  \begin{tikzpicture}[
      node distance=1.2cm and 1.5cm,
      every node/.style={draw, rectangle, rounded corners, align=center, font=\footnotesize},
      thick
  ]
  
  \node (mobile) {React Native\\(iOS/Android)};
  \node[below=0.8cm of mobile] (asyncstorage) [fill=blue!10] {AsyncStorage\\(Локальное хранилище)};
  \node[right=1.5cm of mobile] (api) {API Backend\\(FastAPI)};
  \node[above=0.8cm of api] (jwt) [fill=green!10] {JWT токены};
  \node[below=1.2cm of api] (streaming) [fill=orange!10] {Video Streaming\\(expo-av)};
  \node[below=1.2cm of asyncstorage] (chat) [fill=purple!10] {WebSocket\\(Real-time чат)};
  
  \draw[->,very thick] (mobile) -- node[above]{HTTP(S)} (api);
  \draw[->,very thick] (api) -- (mobile);
  \draw[->,thick] (mobile) -- (asyncstorage);
  \draw[->,thick] (asyncstorage) -- (mobile);
  \draw[->,thick] (jwt) -- (api);
  \draw[->,thick] (api) -- (streaming);
  \draw[->,thick] (streaming) -- (mobile);
  \draw[->,thick] (mobile) -- (chat);
  \draw[->,thick] (chat) -- (api);
  
  \end{tikzpicture}
  }
  \caption{Архитектура мобильного приложения}
\end{figure}

\subsubsection{Технологический стек мобильного приложения}

\begin{itemize}
    \item \textbf{React Native} - кроссплатформенный фреймворк для разработки нативных мобильных приложений на JavaScript
    \item \textbf{Expo} - платформа и набор инструментов для упрощения разработки React Native приложений
    \item \textbf{Expo Router} - файловая маршрутизация для навигации в приложении (аналог Next.js для мобильных)
    \item \textbf{NativeWind} - Tailwind CSS для React Native, позволяющий использовать utility-first подход к стилизации
    \item \textbf{Expo AV} - библиотека для воспроизведения видео и аудио потоков
    \item \textbf{AsyncStorage} - асинхронное хранилище данных для сохранения токенов и настроек пользователя
    \item \textbf{Axios} - HTTP-клиент для взаимодействия с backend API
\end{itemize}

\subsubsection{Структура приложения}

Приложение организовано по принципу файловой маршрутизации Expo Router:

\begin{minted}[fontsize=\footnotesize]{text}
app/
├── (auth)/              # Экраны авторизации
│   ├── login.tsx       # Экран входа
│   └── register.tsx    # Экран регистрации
├── (tabs)/             # Главные экраны с навигацией
│   ├── _layout.tsx     # Layout для tab navigation
│   ├── index.tsx       # Список стримов (главная)
│   └── create.tsx      # Создание стрима
├── stream/
│   └── [id].tsx        # Просмотр стрима с чатом
├── _layout.tsx         # Корневой layout
└── index.tsx           # Точка входа с проверкой авторизации
\end{minted}

\subsubsection{Ключевые особенности реализации}

\paragraph{1. Авторизация и безопасность}

Мобильное приложение использует тот же механизм авторизации, что и веб-версия - JWT токены. Однако вместо \texttt{localStorage} (недоступного в React Native) используется \textbf{AsyncStorage}:

\begin{minted}[fontsize=\footnotesize]{typescript}
// Сохранение токена после успешной авторизации
await AsyncStorage.setItem('authToken', loginData.access_token);
await AsyncStorage.setItem('user', JSON.stringify({ username, email }));

// Получение токена для запросов
const token = await AsyncStorage.getItem('authToken');
if (token) {
    config.headers.Authorization = `Bearer ${token}`;
}
\end{minted}

\paragraph{2. Взаимодействие с Backend}

Для подключения мобильного приложения к backend используется локальный IP-адрес компьютера в сети, а не \texttt{localhost}:

\begin{minted}[fontsize=\footnotesize]{typescript}
// Конфигурация API
export const API_BASE_URL = process.env.EXPO_PUBLIC_API_BASE_URL 
    || 'http://192.168.0.103:8000';

// Для Android эмулятора используется специальный адрес:
// http://10.0.2.2:8000 (указывает на localhost хост-машины)
\end{minted}

Это критически важно, так как \texttt{0.0.0.0} или \texttt{localhost} в мобильном приложении указывает на само устройство, а не на компьютер разработчика.

\paragraph{3. Tab Navigation}

Приложение использует нижнюю tab-навигацию с двумя основными экранами:

\begin{itemize}
    \item \textbf{Стримы} - главная страница со списком активных трансляций
    \item \textbf{Создать} - страница создания нового стрима (требует авторизации)
\end{itemize}

\begin{minted}[fontsize=\footnotesize]{typescript}
<Tabs
  screenOptions={{
    tabBarActiveTintColor: '#9333EA',    // Фиолетовый цвет
    tabBarInactiveTintColor: '#6B7280', // Серый цвет
    tabBarStyle: {
      backgroundColor: '#1F2937',        // Темный фон
    },
  }}
>
  <Tabs.Screen name="index" options={{ title: 'Стримы' }} />
  <Tabs.Screen name="create" options={{ title: 'Создать' }} />
</Tabs>
\end{minted}

\paragraph{4. Экран просмотра стрима}

Экран просмотра включает:

\begin{itemize}
    \item Видеоплеер с использованием \texttt{expo-av}
    \item Информацию о стриме (название, автор, количество зрителей)
    \item Real-time чат с возможностью отправки сообщений
    \item Индикатор "LIVE" для активных трансляций
\end{itemize}

\begin{minted}[fontsize=\footnotesize]{typescript}
<Video
  ref={videoRef}
  source={{ uri: stream.streamUrl }}
  useNativeControls
  resizeMode={ResizeMode.CONTAIN}
  shouldPlay
  onError={(error) => {
    Alert.alert('Ошибка', 'Не удалось загрузить видео');
  }}
/>
\end{minted}

Чат обновляется каждые 3 секунды через polling (в будущем планируется переход на WebSocket):

\begin{minted}[fontsize=\footnotesize]{typescript}
useEffect(() => {
    loadMessages();
    const interval = setInterval(loadMessages, 3000);
    return () => clearInterval(interval);
}, [streamId]);
\end{minted}

\subsubsection{Стилизация с NativeWind}

Приложение использует NativeWind для стилизации - это Tailwind CSS адаптированный для React Native:

\begin{minted}[fontsize=\footnotesize]{typescript}
<View className="flex-1 bg-gray-900 justify-center px-6">
  <Text className="text-4xl font-bold text-white text-center mb-8">
    Регистрация
  </Text>
  <TextInput
    className="bg-gray-800 text-white px-4 py-3 rounded-lg"
    placeholder="Email"
    placeholderTextColor="#9CA3AF"
  />
  <TouchableOpacity className="bg-purple-600 py-4 rounded-lg">
    <Text className="text-white text-center font-semibold">
      Зарегистрироваться
    </Text>
  </TouchableOpacity>
</View>
\end{minted}

Это обеспечивает:
\begin{itemize}
    \item Единый стиль с веб-версией
    \item Быструю разработку UI
    \item Легкую поддержку темной темы
\end{itemize}

\subsubsection{Процесс регистрации в мобильном приложении}

\begin{enumerate}
    \item Пользователь заполняет форму регистрации (username, email, password)
    \item Данные отправляются на \texttt{/auth/register} endpoint
    \item Backend создает пользователя в PostgreSQL
    \item Автоматически выполняется вход через \texttt{/auth/jwt/login}
    \item JWT токен сохраняется в AsyncStorage
    \item Пользователь перенаправляется на главный экран
\end{enumerate}

\begin{minted}[fontsize=\footnotesize]{typescript}
const handleRegister = async () => {
    // 1. Регистрация
    const regResponse = await fetch(`${API_BASE_URL}/auth/register`, {
        method: "POST",
        body: JSON.stringify({ username, email, password })
    });
    
    // 2. Автоматический логин
    const formData = new URLSearchParams();
    formData.append("username", email);
    formData.append("password", password);
    
    const loginResponse = await fetch(`${API_BASE_URL}/auth/jwt/login`, {
        method: "POST",
        body: formData.toString()
    });
    
    // 3. Сохранение токена
    const loginData = await loginResponse.json();
    await AsyncStorage.setItem("access_token", loginData.access_token);
    
    // 4. Навигация
    router.replace('/(tabs)');
};
\end{minted}

\subsubsection{Особенности тестирования мобильного приложения}

Для тестирования мобильного приложения существует несколько способов:

\paragraph{1. Expo Go (рекомендуется для разработки)}
\begin{itemize}
    \item Установить Expo Go на телефон из App Store или Google Play
    \item Запустить \texttt{bun start} на компьютере
    \item Отсканировать QR-код через Expo Go
    \item Телефон и компьютер должны быть в одной Wi-Fi сети
\end{itemize}

\paragraph{2. Android эмулятор}
\begin{itemize}
    \item Установить Android Studio
    \item Создать виртуальное устройство (AVD)
    \item Запустить \texttt{bun run android}
    \item Использовать \texttt{http://10.0.2.2:8000} для обращения к localhost
\end{itemize}

\paragraph{3. iOS симулятор (только на macOS)}
\begin{itemize}
    \item Установить Xcode
    \item Запустить \texttt{bun run ios}
    \item Использовать \texttt{http://localhost:8000} для обращения к backend
\end{itemize}

\subsubsection{Проблемы и решения}

\paragraph{Проблема: Network Error при запросах}
\begin{itemize}
    \item \textbf{Причина}: Использование \texttt{localhost} или \texttt{0.0.0.0}
    \item \textbf{Решение}: Использовать локальный IP (например, \texttt{192.168.0.103})
    \item \textbf{Как узнать IP}: \texttt{ip addr show | grep "inet "}
\end{itemize}

\paragraph{Проблема: localStorage is not defined}
\begin{itemize}
    \item \textbf{Причина}: \texttt{localStorage} недоступен в React Native
    \item \textbf{Решение}: Использовать \texttt{AsyncStorage} вместо \texttt{localStorage}
\end{itemize}

\paragraph{Проблема: Navigation before mounting}
\begin{itemize}
    \item \textbf{Причина}: Попытка навигации до полной загрузки Root Layout
    \item \textbf{Решение}: Использовать \texttt{setTimeout} с небольшой задержкой (100ms)
\end{itemize}

\subsubsection{Планы развития мобильного приложения}

\begin{enumerate}
    \item \textbf{Push-уведомления}
    \begin{itemize}
        \item Уведомления о начале стримов подписок
        \item Уведомления о новых сообщениях в чате
    \end{itemize}
    
    \item \textbf{Улучшение видеоплеера}
    \begin{itemize}
        \item Поддержка различных качеств видео
        \item Picture-in-Picture режим
        \item Landscape ориентация для полноэкранного просмотра
    \end{itemize}
    
    \item \textbf{WebSocket для чата}
    \begin{itemize}
        \item Замена polling на real-time соединение
        \item Уменьшение задержки сообщений
        \item Снижение нагрузки на сервер
    \end{itemize}
    
    \item \textbf{Офлайн режим}
    \begin{itemize}
        \item Кеширование списка стримов
        \item Возможность просмотра сохраненных трансляций
    \end{itemize}
    
    \item \textbf{Production build}
    \begin{itemize}
        \item Подготовка для публикации в App Store и Google Play
        \item Использование EAS Build для создания нативных сборок
        \item Настройка CI/CD для автоматического деплоя
    \end{itemize}
\end{enumerate}

\subsubsection{Выводы по мобильному приложению}

Мобильное приложение Leams демонстрирует возможность создания полнофункционального кроссплатформенного решения с использованием современного стека технологий. Использование React Native и Expo позволяет:

\begin{itemize}
    \item Разрабатывать одновременно для iOS и Android
    \item Переиспользовать логику с веб-версии
    \item Быстро итерироваться и тестировать изменения
    \item Обеспечить единый пользовательский опыт на всех платформах
\end{itemize}

Интеграция с существующим FastAPI backend через REST API обеспечивает надежное и безопасное взаимодействие, а использование JWT токенов гарантирует защиту пользовательских данных.
